% Use XeLaTeX
% Overleaf project URL : https://www.overleaf.com/project/631155904c1a625604a6ea03

\documentclass{article}

\usepackage{arxiv}

% \usepackage[french]{babel}    % Décommenter pour écrire en français
\usepackage[T1]{fontenc}      % use 8-bit T1 fonts
\usepackage{hyperref}         % hyperlinks
\usepackage{url}              % simple URL typesetting
\usepackage{booktabs}         % professional-quality tables
\usepackage{amssymb}          % blackboard math symbols
\usepackage{pifont}           % Ballot symbols
\usepackage{fontspec}         % XeLaTeX
\usepackage{nicefrac}         % compact symbols for 1/2, etc.
\usepackage{microtype}        % microtypography
\usepackage{lipsum}
\usepackage{float}            % inline table with [H]
\usepackage{graphicx}         % show image

\setmainfont{Times New Roman} % XeLaTeX

\newcommand{\cmark}{\ding{51}} % Check mark
\newcommand{\xmark}{\ding{55}} % X mark

\graphicspath{ {./} }

\title{
  Fiber Config Generator \\
  An attempt at creating a 3D white matter fiber crossing simulator using preexisting tools}


\author{
  Benoît Dubreuil \\
  Computer Science department \\
  Université du Québec à Montréal \\
  Montréal, Canada H3C 3P8 \\
  \texttt{dubreuil.benoit.2@courrier.uqam.ca} \\
  \And
  Joël Lefebvre \\
  Computer Science department\\
  Université du Québec à Montréal \\
  Montréal, Canada H3C 3P8\\
  \texttt{lefebvre.joel@uqam.ca} \\
}

\begin{document}
  \maketitle

  \begin{abstract}
    TODO: the paper abstract in 300--400 words.
  \end{abstract}


  \section{Introduction}\label{sec:introduction}

  The goal of the project Fiber Config Generator is to create a 3D white matter fiber crossing simulator to generate synthetic data in order to validate our new algorithms for processing of light sheet microscopy data.
  To do so, the project was divided into two parts : identify potential libraries to delegate the white matter phantom
  generation to and single out one according to predefined selection criteria (1), and develop an application that utilizes the selected library in order to generate high-level fiber configurations (2).
  The first part was achieved in the course ``initiation to
  research'' that preceded this internship, back in the winter 2022 academic term~\cite{dubreuil2022inf6200}.
  The \href{https://github.com/linum-uqam/inm5803-ete2022-benoit-dubreuil/}{second part} is the objective of this internship, which takes place in the summer 2022 academic term.


  \section{Methods}\label{sec:methodology}

  \subsection{Part 1: Identify and select a white matter generation library}\label{subsec:part-1}

  TODO

  \subsection{Part 2: Develop an application that utilizes the selected library}\label{subsec:part-2}

  TODO


  \section{Results}\label{sec:results}

  TODO


  \section{Conclusion}\label{sec:conclusion}

  In conclusion, in order to create a 3D white matter fiber crossing simulator to generate synthetic data, the usage of external MRI-centric tools with white matter generation features to circumvent the absence of fiber bundle generation-centric software (M. Descoteaux \& A. Valcourt Caron, virtual conversation, 21 March 2022) ought to be avoided.
  Accordingly, the development of a novel application program is recommended.


  \section*{Acknowledgments}\label{sec:thanks}

  I would like to thank my research professor-supervisor Joël Lefebvre who took me under his wing and trained me in the fields of image processing and neuroimaging.


  % Ajoutez vos références (format bibtex) dans le fichier references.bib
  \bibliographystyle{unsrturl}
  \bibliography{references}

\end{document}
