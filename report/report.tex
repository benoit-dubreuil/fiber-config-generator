% Use XeLaTeX
% Overleaf project URL : https://www.overleaf.com/project/631155904c1a625604a6ea03

\documentclass{article}

\usepackage{arxiv}

% \usepackage[french]{babel}    % Décommenter pour écrire en français
\usepackage[T1]{fontenc}      % use 8-bit T1 fonts
\usepackage{hyperref}         % hyperlinks
\usepackage{url}              % simple URL typesetting
\usepackage{booktabs}         % professional-quality tables
\usepackage{amssymb}          % blackboard math symbols
\usepackage{pifont}           % Ballot symbols
\usepackage{fontspec}         % XeLaTeX
\usepackage{nicefrac}         % compact symbols for 1/2, etc.
\usepackage{microtype}        % microtypography
\usepackage{lipsum}
\usepackage{float}            % inline table with [H]
\usepackage{graphicx}         % show image

\setmainfont{Times New Roman} % XeLaTeX

\newcommand{\cmark}{\ding{51}} % Check mark
\newcommand{\xmark}{\ding{55}} % X mark

\graphicspath{ {./} }

\title{
  Final Report \\
  INM5803, Initiation à la recherche scientifique}


\author{
  Benoît Dubreuil \\
  Computer Science department \\
  Université du Québec à Montréal \\
  Montréal, Canada H3C 3P8 \\
  \texttt{dubreuil.benoit.2@courrier.uqam.ca} \\
  \And
  Joël Lefebvre \\
  Computer Science department\\
  Université du Québec à Montréal \\
  Montréal, Canada H3C 3P8\\
  \texttt{lefebvre.joel@uqam.ca} \\
}

\begin{document}
  \maketitle

  \begin{abstract}
    TODO: the paper abstract in 300--400 words.
  \end{abstract}


  \section{Introduction}
  \label{sec:introduction}

  TODO


  \section{Methods}
  \label{sec:methodology}

  TODO


  \section{Results}
  \label{sec:results}

  TODO


  \section{Conclusion}
  \label{sec:conclusion}

  TODO


  \section*{Acknowledgments}
  \label{sec:thanks}

  TODO

  Je souhaite remercier mon professeur-superviseur de recherche Joël Lefebvre qui m'a pris sous son aile et qui m'a formé sur les domaines du traitement d'images et
  de la neuroimagerie.


  % Ajoutez vos références (format bibtex) dans le fichier references.bib
  \bibliographystyle{unsrt}
  \bibliography{references}

\end{document}
